\documentclass[]{article}
\usepackage{lmodern}
\usepackage{amssymb,amsmath}
\usepackage{ifxetex,ifluatex}
\usepackage{fixltx2e} % provides \textsubscript
\ifnum 0\ifxetex 1\fi\ifluatex 1\fi=0 % if pdftex
  \usepackage[T1]{fontenc}
  \usepackage[utf8]{inputenc}
\else % if luatex or xelatex
  \ifxetex
    \usepackage{mathspec}
  \else
    \usepackage{fontspec}
  \fi
  \defaultfontfeatures{Ligatures=TeX,Scale=MatchLowercase}
\fi
% use upquote if available, for straight quotes in verbatim environments
\IfFileExists{upquote.sty}{\usepackage{upquote}}{}
% use microtype if available
\IfFileExists{microtype.sty}{%
\usepackage[]{microtype}
\UseMicrotypeSet[protrusion]{basicmath} % disable protrusion for tt fonts
}{}
\PassOptionsToPackage{hyphens}{url} % url is loaded by hyperref
\usepackage[unicode=true]{hyperref}
\hypersetup{
            pdftitle={Conditional Copula},
            pdfborder={0 0 0},
            breaklinks=true}
\urlstyle{same}  % don't use monospace font for urls
\usepackage[margin=1in]{geometry}
\usepackage{color}
\usepackage{fancyvrb}
\newcommand{\VerbBar}{|}
\newcommand{\VERB}{\Verb[commandchars=\\\{\}]}
\DefineVerbatimEnvironment{Highlighting}{Verbatim}{commandchars=\\\{\}}
% Add ',fontsize=\small' for more characters per line
\usepackage{framed}
\definecolor{shadecolor}{RGB}{248,248,248}
\newenvironment{Shaded}{\begin{snugshade}}{\end{snugshade}}
\newcommand{\KeywordTok}[1]{\textcolor[rgb]{0.13,0.29,0.53}{\textbf{#1}}}
\newcommand{\DataTypeTok}[1]{\textcolor[rgb]{0.13,0.29,0.53}{#1}}
\newcommand{\DecValTok}[1]{\textcolor[rgb]{0.00,0.00,0.81}{#1}}
\newcommand{\BaseNTok}[1]{\textcolor[rgb]{0.00,0.00,0.81}{#1}}
\newcommand{\FloatTok}[1]{\textcolor[rgb]{0.00,0.00,0.81}{#1}}
\newcommand{\ConstantTok}[1]{\textcolor[rgb]{0.00,0.00,0.00}{#1}}
\newcommand{\CharTok}[1]{\textcolor[rgb]{0.31,0.60,0.02}{#1}}
\newcommand{\SpecialCharTok}[1]{\textcolor[rgb]{0.00,0.00,0.00}{#1}}
\newcommand{\StringTok}[1]{\textcolor[rgb]{0.31,0.60,0.02}{#1}}
\newcommand{\VerbatimStringTok}[1]{\textcolor[rgb]{0.31,0.60,0.02}{#1}}
\newcommand{\SpecialStringTok}[1]{\textcolor[rgb]{0.31,0.60,0.02}{#1}}
\newcommand{\ImportTok}[1]{#1}
\newcommand{\CommentTok}[1]{\textcolor[rgb]{0.56,0.35,0.01}{\textit{#1}}}
\newcommand{\DocumentationTok}[1]{\textcolor[rgb]{0.56,0.35,0.01}{\textbf{\textit{#1}}}}
\newcommand{\AnnotationTok}[1]{\textcolor[rgb]{0.56,0.35,0.01}{\textbf{\textit{#1}}}}
\newcommand{\CommentVarTok}[1]{\textcolor[rgb]{0.56,0.35,0.01}{\textbf{\textit{#1}}}}
\newcommand{\OtherTok}[1]{\textcolor[rgb]{0.56,0.35,0.01}{#1}}
\newcommand{\FunctionTok}[1]{\textcolor[rgb]{0.00,0.00,0.00}{#1}}
\newcommand{\VariableTok}[1]{\textcolor[rgb]{0.00,0.00,0.00}{#1}}
\newcommand{\ControlFlowTok}[1]{\textcolor[rgb]{0.13,0.29,0.53}{\textbf{#1}}}
\newcommand{\OperatorTok}[1]{\textcolor[rgb]{0.81,0.36,0.00}{\textbf{#1}}}
\newcommand{\BuiltInTok}[1]{#1}
\newcommand{\ExtensionTok}[1]{#1}
\newcommand{\PreprocessorTok}[1]{\textcolor[rgb]{0.56,0.35,0.01}{\textit{#1}}}
\newcommand{\AttributeTok}[1]{\textcolor[rgb]{0.77,0.63,0.00}{#1}}
\newcommand{\RegionMarkerTok}[1]{#1}
\newcommand{\InformationTok}[1]{\textcolor[rgb]{0.56,0.35,0.01}{\textbf{\textit{#1}}}}
\newcommand{\WarningTok}[1]{\textcolor[rgb]{0.56,0.35,0.01}{\textbf{\textit{#1}}}}
\newcommand{\AlertTok}[1]{\textcolor[rgb]{0.94,0.16,0.16}{#1}}
\newcommand{\ErrorTok}[1]{\textcolor[rgb]{0.64,0.00,0.00}{\textbf{#1}}}
\newcommand{\NormalTok}[1]{#1}
\usepackage{graphicx,grffile}
\makeatletter
\def\maxwidth{\ifdim\Gin@nat@width>\linewidth\linewidth\else\Gin@nat@width\fi}
\def\maxheight{\ifdim\Gin@nat@height>\textheight\textheight\else\Gin@nat@height\fi}
\makeatother
% Scale images if necessary, so that they will not overflow the page
% margins by default, and it is still possible to overwrite the defaults
% using explicit options in \includegraphics[width, height, ...]{}
\setkeys{Gin}{width=\maxwidth,height=\maxheight,keepaspectratio}
\IfFileExists{parskip.sty}{%
\usepackage{parskip}
}{% else
\setlength{\parindent}{0pt}
\setlength{\parskip}{6pt plus 2pt minus 1pt}
}
\setlength{\emergencystretch}{3em}  % prevent overfull lines
\providecommand{\tightlist}{%
  \setlength{\itemsep}{0pt}\setlength{\parskip}{0pt}}
\setcounter{secnumdepth}{0}
% Redefines (sub)paragraphs to behave more like sections
\ifx\paragraph\undefined\else
\let\oldparagraph\paragraph
\renewcommand{\paragraph}[1]{\oldparagraph{#1}\mbox{}}
\fi
\ifx\subparagraph\undefined\else
\let\oldsubparagraph\subparagraph
\renewcommand{\subparagraph}[1]{\oldsubparagraph{#1}\mbox{}}
\fi

% set default figure placement to htbp
\makeatletter
\def\fps@figure{htbp}
\makeatother


\title{Conditional Copula}
\author{}
\date{\vspace{-2.5em}}

\begin{document}
\maketitle

\subsection{Example of a Conditional
Copula}\label{example-of-a-conditional-copula}

\def\mb \#1\{\boldsymbol{#1}\}

Consider the two-dimensional random vector \((X, Y)\) having joint
conditional distribution \[
f(x,y|z)=f_X(x)f_Y(y)c(F_X(x),F_Y(y)\mid z)
\] where each marginal is a mixture of univariate normal random
variables. \[
f_X(x) = f_Y(x) = \sum_{r=1}^{R}\pi_r\phi(x\mid \mu_r, \sigma_r)
\] We chose the following settings and parameter values: \(f_X = f_Y\),
\(R = 6\), \(\mu_{1:6} = (-9, -5.4, -1.8, 1.8, 5.4, 9)\), and
\(\sigma_r = 1/\sqrt{10}\), \(\pi_r = 1/R\).

and the copula is \[
C(u,v \mid x )=u^{1-\alpha(x)}v^{1-\beta(x)}[u^{-\theta\alpha(x)}+v^{-\theta\beta(x)}-1]^{-1/\theta}
\]

where

\begin{itemize}
\item
  \(\theta=20\)
\item
  \[
    \alpha(x)=\frac{2}{3}-\frac{1}{4}\times\frac{1}{\exp(x)+1}
    \]
\item
  \[
    \beta(x)=\frac{1}{2}+\frac{1}{4}\times\frac{\exp(x)}{\exp(x)+1}
    \]
\end{itemize}

Draw a sample when \(x\in (-4,4)\)

\begin{Shaded}
\begin{Highlighting}[]
\KeywordTok{set.seed}\NormalTok{(}\DecValTok{100}\NormalTok{)}

\NormalTok{R=}\DecValTok{6}
\NormalTok{mu=}\KeywordTok{c}\NormalTok{(}\OperatorTok{-}\DecValTok{9}\NormalTok{, }\OperatorTok{-}\FloatTok{5.4}\NormalTok{, }\OperatorTok{-}\FloatTok{1.8}\NormalTok{, }\FloatTok{1.8}\NormalTok{, }\FloatTok{5.4}\NormalTok{, }\DecValTok{9}\NormalTok{)}
\NormalTok{sigmar=}\KeywordTok{rep}\NormalTok{(}\DecValTok{1}\OperatorTok{/}\KeywordTok{sqrt}\NormalTok{(}\DecValTok{10}\NormalTok{), R)}
\NormalTok{pi=}\KeywordTok{rep}\NormalTok{(}\DecValTok{1}\OperatorTok{/}\NormalTok{R, R)}
\NormalTok{theta=}\DecValTok{40}

\CommentTok{# evaluate the function at the point x, where the components }
\CommentTok{# of the mixture have weights w, means stored in u, and std deviations}
\CommentTok{# stored in s - all must have the same length.}
\NormalTok{F =}\StringTok{ }\ControlFlowTok{function}\NormalTok{(x,w,u,s) }\KeywordTok{sum}\NormalTok{( w}\OperatorTok{*}\KeywordTok{pnorm}\NormalTok{(x,}\DataTypeTok{mean=}\NormalTok{u,}\DataTypeTok{sd=}\NormalTok{s) )}

\CommentTok{# provide an initial bracket for the quantile. default is c(-1000,1000). }
\NormalTok{F_inv =}\StringTok{ }\ControlFlowTok{function}\NormalTok{(p,w,u,s,}\DataTypeTok{br=}\KeywordTok{c}\NormalTok{(}\OperatorTok{-}\DecValTok{1000}\NormalTok{,}\DecValTok{1000}\NormalTok{))}
\NormalTok{\{}
\NormalTok{  G =}\StringTok{ }\ControlFlowTok{function}\NormalTok{(x) }\KeywordTok{F}\NormalTok{(x,w,u,s) }\OperatorTok{-}\StringTok{ }\NormalTok{p}
  \KeywordTok{return}\NormalTok{( }\KeywordTok{uniroot}\NormalTok{(G,br)}\OperatorTok{$}\NormalTok{root ) }
\NormalTok{\}}


\NormalTok{l=}\DecValTok{5}
\CommentTok{#function alpha(x)}
\NormalTok{a<-}\ControlFlowTok{function}\NormalTok{(x)\{}
  \DecValTok{2}\OperatorTok{/}\DecValTok{3}\OperatorTok{-}\DecValTok{1}\OperatorTok{/}\DecValTok{4}\OperatorTok{*}\NormalTok{(}\KeywordTok{exp}\NormalTok{(l}\OperatorTok{*}\NormalTok{x)}\OperatorTok{/}\NormalTok{(}\KeywordTok{exp}\NormalTok{(l}\OperatorTok{*}\NormalTok{x)}\OperatorTok{+}\DecValTok{1}\NormalTok{))}
\NormalTok{\}}
\CommentTok{#function beta(x)}
\NormalTok{b<-}\ControlFlowTok{function}\NormalTok{(x)\{}
  \DecValTok{1}\OperatorTok{/}\DecValTok{2}\OperatorTok{+}\DecValTok{1}\OperatorTok{/}\DecValTok{4}\OperatorTok{*}\NormalTok{(}\KeywordTok{exp}\NormalTok{(l}\OperatorTok{*}\NormalTok{x)}\OperatorTok{/}\NormalTok{(}\KeywordTok{exp}\NormalTok{(l}\OperatorTok{*}\NormalTok{x)}\OperatorTok{+}\DecValTok{1}\NormalTok{))}
\NormalTok{\}}

\NormalTok{n=}\DecValTok{1500}
\NormalTok{V=}\KeywordTok{runif}\NormalTok{(n)}
\NormalTok{W=}\KeywordTok{runif}\NormalTok{(n)}

\NormalTok{X=}\KeywordTok{seq}\NormalTok{(}\OperatorTok{-}\DecValTok{4}\NormalTok{,}\DecValTok{4}\NormalTok{, }\DataTypeTok{length.out =}\NormalTok{ n)}

\NormalTok{C <-}\StringTok{ }\ControlFlowTok{function}\NormalTok{(u) \{}
\NormalTok{  (}\DecValTok{1}\OperatorTok{-}\NormalTok{beta)}\OperatorTok{*}\NormalTok{u}\OperatorTok{^}\NormalTok{(}\DecValTok{1}\OperatorTok{-}\NormalTok{alpha)}\OperatorTok{*}\NormalTok{v}\OperatorTok{^}\NormalTok{(}\OperatorTok{-}\NormalTok{beta)}\OperatorTok{*}\NormalTok{(u}\OperatorTok{^}\NormalTok{(}\OperatorTok{-}\NormalTok{theta}\OperatorTok{*}\NormalTok{alpha)}\OperatorTok{+}\NormalTok{v}\OperatorTok{^}\NormalTok{(}\OperatorTok{-}\NormalTok{theta}\OperatorTok{*}\NormalTok{beta)}\OperatorTok{-}\DecValTok{1}\NormalTok{)}\OperatorTok{^}\NormalTok{(}\OperatorTok{-}\DecValTok{1}\OperatorTok{/}\NormalTok{theta)}\OperatorTok{+}
\StringTok{    }\NormalTok{beta}\OperatorTok{*}\NormalTok{u}\OperatorTok{^}\NormalTok{(}\DecValTok{1}\OperatorTok{-}\NormalTok{alpha)}\OperatorTok{*}\NormalTok{v}\OperatorTok{^}\NormalTok{(}\OperatorTok{-}\NormalTok{beta}\OperatorTok{*}\NormalTok{(}\DecValTok{1}\OperatorTok{+}\NormalTok{theta))}\OperatorTok{*}\NormalTok{(u}\OperatorTok{^}\NormalTok{(}\OperatorTok{-}\NormalTok{theta}\OperatorTok{*}\NormalTok{alpha)}\OperatorTok{+}\NormalTok{v}\OperatorTok{^}\NormalTok{(}\OperatorTok{-}\NormalTok{theta}\OperatorTok{*}\NormalTok{beta)}\OperatorTok{-}\DecValTok{1}\NormalTok{)}\OperatorTok{^}\NormalTok{(}\OperatorTok{-}\DecValTok{1}\OperatorTok{/}\NormalTok{theta}\OperatorTok{-}\DecValTok{1}\NormalTok{)}\OperatorTok{-}\NormalTok{w}
\NormalTok{\}}

\NormalTok{U=}\KeywordTok{rep}\NormalTok{(}\DecValTok{0}\NormalTok{,n)}
\ControlFlowTok{for}\NormalTok{ (i }\ControlFlowTok{in} \DecValTok{1}\OperatorTok{:}\NormalTok{n) \{}
\NormalTok{  v=V[i]}
\NormalTok{  w=W[i]}
\NormalTok{  alpha=}\KeywordTok{a}\NormalTok{(X[i])}
\NormalTok{  beta=}\KeywordTok{b}\NormalTok{(X[i])}
\NormalTok{  U[i]=}\KeywordTok{uniroot}\NormalTok{(C, }\DataTypeTok{lower =} \DecValTok{0}\NormalTok{, }\DataTypeTok{upper =} \DecValTok{1}\NormalTok{)}\OperatorTok{$}\NormalTok{root}
\NormalTok{\}}
\end{Highlighting}
\end{Shaded}

\begin{Shaded}
\begin{Highlighting}[]
\KeywordTok{par}\NormalTok{(}\DataTypeTok{mfrow=}\KeywordTok{c}\NormalTok{(}\DecValTok{1}\NormalTok{,}\DecValTok{3}\NormalTok{))}
\KeywordTok{plot}\NormalTok{(U[}\DecValTok{1}\OperatorTok{:}\NormalTok{(}\DecValTok{1}\OperatorTok{/}\DecValTok{3}\OperatorTok{*}\NormalTok{n)], V[}\DecValTok{1}\OperatorTok{:}\NormalTok{(}\DecValTok{1}\OperatorTok{/}\DecValTok{3}\OperatorTok{*}\NormalTok{n)], }\DataTypeTok{xlab =} \StringTok{'U'}\NormalTok{, }\DataTypeTok{ylab=}\StringTok{'V'}\NormalTok{, }\DataTypeTok{main =} \StringTok{'-4<x<-1.3'}\NormalTok{)}
\KeywordTok{plot}\NormalTok{(U[(}\DecValTok{1}\OperatorTok{/}\DecValTok{3}\OperatorTok{*}\NormalTok{n)}\OperatorTok{:}\NormalTok{(}\DecValTok{2}\OperatorTok{/}\DecValTok{3}\OperatorTok{*}\NormalTok{n)], V[(}\DecValTok{1}\OperatorTok{/}\DecValTok{3}\OperatorTok{*}\NormalTok{n)}\OperatorTok{:}\NormalTok{(}\DecValTok{2}\OperatorTok{/}\DecValTok{3}\OperatorTok{*}\NormalTok{n)],}\DataTypeTok{xlab =} \StringTok{'U'}\NormalTok{, }\DataTypeTok{ylab=}\StringTok{'V'}\NormalTok{, }\DataTypeTok{main =} \StringTok{'-1.3<x<1.3'}\NormalTok{)}
\KeywordTok{plot}\NormalTok{(U[(}\DecValTok{2}\OperatorTok{/}\DecValTok{3}\OperatorTok{*}\NormalTok{n)}\OperatorTok{:}\NormalTok{n], V[(}\DecValTok{2}\OperatorTok{/}\DecValTok{3}\OperatorTok{*}\NormalTok{n)}\OperatorTok{:}\NormalTok{n], }\DataTypeTok{xlab =} \StringTok{'U'}\NormalTok{, }\DataTypeTok{ylab=}\StringTok{'V'}\NormalTok{, }\DataTypeTok{main =} \StringTok{'1.3<x<4'}\NormalTok{)}
\end{Highlighting}
\end{Shaded}

\includegraphics{Conditional-Copula_files/figure-latex/unnamed-chunk-3-1.pdf}

\begin{Shaded}
\begin{Highlighting}[]
\NormalTok{X1=}\KeywordTok{sapply}\NormalTok{(V, F_inv, }\DataTypeTok{w=}\NormalTok{pi, }\DataTypeTok{u=}\NormalTok{mu, }\DataTypeTok{s=}\NormalTok{sigmar)}
\NormalTok{X2=}\KeywordTok{sapply}\NormalTok{(U, F_inv, }\DataTypeTok{w=}\NormalTok{pi, }\DataTypeTok{u=}\NormalTok{mu, }\DataTypeTok{s=}\NormalTok{sigmar)}

\KeywordTok{par}\NormalTok{(}\DataTypeTok{mfrow=}\KeywordTok{c}\NormalTok{(}\DecValTok{1}\NormalTok{,}\DecValTok{3}\NormalTok{))}
\KeywordTok{plot}\NormalTok{(X1[}\DecValTok{1}\OperatorTok{:}\NormalTok{(}\DecValTok{1}\OperatorTok{/}\DecValTok{3}\OperatorTok{*}\NormalTok{n)], X2[}\DecValTok{1}\OperatorTok{:}\NormalTok{(}\DecValTok{1}\OperatorTok{/}\DecValTok{3}\OperatorTok{*}\NormalTok{n)],  }\DataTypeTok{xlab =} \StringTok{'X'}\NormalTok{, }\DataTypeTok{ylab=}\StringTok{'Y'}\NormalTok{, }\DataTypeTok{main =}\StringTok{'-4<x<-1.3'}\NormalTok{)}
\KeywordTok{plot}\NormalTok{(X1[(}\DecValTok{1}\OperatorTok{/}\DecValTok{3}\OperatorTok{*}\NormalTok{n)}\OperatorTok{:}\NormalTok{(}\DecValTok{2}\OperatorTok{/}\DecValTok{3}\OperatorTok{*}\NormalTok{n)], X2[(}\DecValTok{1}\OperatorTok{/}\DecValTok{3}\OperatorTok{*}\NormalTok{n)}\OperatorTok{:}\NormalTok{(}\DecValTok{2}\OperatorTok{/}\DecValTok{3}\OperatorTok{*}\NormalTok{n)], }\DataTypeTok{xlab =} \StringTok{'X'}\NormalTok{, }\DataTypeTok{ylab=}\StringTok{'Y'}\NormalTok{, }\DataTypeTok{main =}  \StringTok{'-1.3<x<1.3'}\NormalTok{)}
\KeywordTok{plot}\NormalTok{(X1[(}\DecValTok{2}\OperatorTok{/}\DecValTok{3}\OperatorTok{*}\NormalTok{n)}\OperatorTok{:}\NormalTok{n], X2[(}\DecValTok{2}\OperatorTok{/}\DecValTok{3}\OperatorTok{*}\NormalTok{n)}\OperatorTok{:}\NormalTok{n], }\DataTypeTok{xlab =} \StringTok{'X'}\NormalTok{, }\DataTypeTok{ylab=}\StringTok{'Y'}\NormalTok{, }\DataTypeTok{main =} \StringTok{'1.3<x<4'}\NormalTok{)}
\end{Highlighting}
\end{Shaded}

\includegraphics{Conditional-Copula_files/figure-latex/unnamed-chunk-4-1.pdf}

\section{Another Example of Conditional
Copula}\label{another-example-of-conditional-copula}

Notation:

\begin{itemize}
\tightlist
\item
  \(\Phi\) is the cdf of the standard normal distribution.
\item
  \(\phi\) is the pdf of standard normal distribution.
\item
  \(\phi_2\) is the pdf of bivariate normal.
\end{itemize}

Then they have \[
p(u_1,u_2 \mid \mu, \sigma)=\frac{\phi_2(\Phi^{-1}(u_1), \Phi^{-1}(u_2)  \mid\mb\mu, \sigma I_2)}{\phi(\Phi^{-1}(u_1))\phi(\Phi^{-1}(u_2))}
\] \textbf{A mixture of \(p(\mb u)\) can be used to estimate any
arbitrarily continuous density on \((0,1)^2\).} \[
\tilde{c}(u_1,u_2 \mid \mb \pi, \mb \mu, \mb \sigma) =\sum^J_{j=1} \pi_j \frac{\phi_2(\Phi^{-1}(u_1),\Phi^{-1}(u_2)  \mid \mb\mu_j, \sigma_j I_2)}{\phi(\Phi^{-1}(u_1))\phi(\Phi^{-1}(u_2))}
\] where \(\pi_j\)s are mixing proportions.

\textbf{When \(\pi_j\) and \(\mu_j\) are conditioned on covariates:}

\begin{itemize}
\item
  \[
    \pi_j(x_i)=\frac{\exp(\beta_j^T x_i)}{\sum^J_{j=1} \exp(\beta_j^Tx_i)}\\
    \]
\item
  \[
    \mb\mu_{j}=
    \begin{bmatrix}
    \lambda_{j1}^Tx_i \\
    \lambda_{j2}^Tx_i 
    \end{bmatrix}
    \]
\end{itemize}

We give an example of a mixture with two components

\begin{itemize}
\item
  \[
    \beta=
    \begin{bmatrix}
    0.1 \\
    0.3
    \end{bmatrix}
    \]
\item
  \[
    \lambda_{11}=
    \begin{bmatrix}
    0.1 \\
    0.3
    \end{bmatrix}
   \]
\end{itemize}

\[
  \lambda_{12}=
  \begin{bmatrix}
  0.1 \\
  0.4
  \end{bmatrix}
  \]

\[
  \lambda_{21}=
  \begin{bmatrix}
  0.3\\
  0.1 
  \end{bmatrix}
  \]

\[
  \lambda_{22}=
  \begin{bmatrix}
  0.4\\
  0.1 
  \end{bmatrix}
  \]

\begin{itemize}
\tightlist
\item
  \(\sigma_1=0.5\) and \(\sigma_2=1\).
\end{itemize}

Draw a random sample from above model when \(X\in(0, 5)\).

\begin{Shaded}
\begin{Highlighting}[]
\KeywordTok{set.seed}\NormalTok{(}\DecValTok{100}\NormalTok{)}

\NormalTok{n=}\DecValTok{1000}
\NormalTok{X=}\KeywordTok{seq}\NormalTok{(}\DecValTok{0}\NormalTok{,}\DecValTok{5}\NormalTok{, }\DataTypeTok{length.out =}\NormalTok{ n)}
\NormalTok{X=}\KeywordTok{cbind}\NormalTok{(}\KeywordTok{rep}\NormalTok{(}\DecValTok{1}\NormalTok{, n), X)}
\NormalTok{delta=}\KeywordTok{c}\NormalTok{(}\FloatTok{0.1}\NormalTok{, }\FloatTok{0.3}\NormalTok{)}

\NormalTok{pi=}\KeywordTok{cbind}\NormalTok{(}\KeywordTok{exp}\NormalTok{(X}\OperatorTok\NormalTok{delta)}\OperatorTok{/}\NormalTok{(}\KeywordTok{exp}\NormalTok{(X}\OperatorTok\NormalTok{delta)}\OperatorTok{+}\DecValTok{1}\NormalTok{), }\DecValTok{1}\OperatorTok{/}\NormalTok{(}\KeywordTok{exp}\NormalTok{(X}\OperatorTok\NormalTok{delta)}\OperatorTok{+}\DecValTok{1}\NormalTok{))}
\CommentTok{#Generate "latent" variable indicating which component}
\NormalTok{h=}\KeywordTok{t}\NormalTok{(}\KeywordTok{apply}\NormalTok{(pi, }\DecValTok{1}\NormalTok{,  rmultinom, }\DataTypeTok{n=}\DecValTok{1}\NormalTok{, }\DataTypeTok{size=}\DecValTok{1}\NormalTok{))}


\CommentTok{#The first bivariate normal component}
\NormalTok{beta11=}\KeywordTok{c}\NormalTok{(}\FloatTok{0.1}\NormalTok{,}\FloatTok{0.3}\NormalTok{)}
\NormalTok{beta12=}\KeywordTok{c}\NormalTok{(}\FloatTok{0.1}\NormalTok{,}\FloatTok{0.4}\NormalTok{)}
\NormalTok{sigma1=}\FloatTok{0.5}
\NormalTok{I=}\KeywordTok{matrix}\NormalTok{(}\KeywordTok{c}\NormalTok{(}\DecValTok{1}\NormalTok{,}\DecValTok{0}\NormalTok{,}\DecValTok{0}\NormalTok{,}\DecValTok{1}\NormalTok{), }\DataTypeTok{nrow =} \DecValTok{2}\NormalTok{)}
\NormalTok{mu1=}\KeywordTok{cbind}\NormalTok{((X}\OperatorTok\NormalTok{beta11)[,}\DecValTok{1}\NormalTok{], (X}\OperatorTok\NormalTok{beta12)[,}\DecValTok{1}\NormalTok{])}

\CommentTok{#Sample from the first component}
\NormalTok{Z1=}\KeywordTok{t}\NormalTok{(}\KeywordTok{apply}\NormalTok{(mu1, }\DecValTok{1}\NormalTok{, rmvnorm, }\DataTypeTok{n=}\DecValTok{1}\NormalTok{, }\DataTypeTok{sigma=}\NormalTok{sigma1}\OperatorTok{*}\NormalTok{I))}



\CommentTok{#The second bivariate normal component}
\NormalTok{beta21=}\KeywordTok{c}\NormalTok{(}\FloatTok{0.3}\NormalTok{, }\FloatTok{0.1}\NormalTok{)}
\NormalTok{beta22=}\KeywordTok{c}\NormalTok{(}\FloatTok{0.4}\NormalTok{, }\FloatTok{0.1}\NormalTok{)}
\NormalTok{sigma2=}\DecValTok{1}
\NormalTok{I=}\KeywordTok{matrix}\NormalTok{(}\KeywordTok{c}\NormalTok{(}\DecValTok{1}\NormalTok{,}\DecValTok{0}\NormalTok{,}\DecValTok{0}\NormalTok{,}\DecValTok{1}\NormalTok{), }\DataTypeTok{nrow =} \DecValTok{2}\NormalTok{)}
\CommentTok{#mean vector}
\NormalTok{mu2=}\KeywordTok{cbind}\NormalTok{((X}\OperatorTok\NormalTok{beta21)[,}\DecValTok{1}\NormalTok{], (X}\OperatorTok\NormalTok{beta22)[,}\DecValTok{1}\NormalTok{])}
\CommentTok{#sample from the second component}
\NormalTok{Z2=}\KeywordTok{t}\NormalTok{(}\KeywordTok{apply}\NormalTok{(mu2, }\DecValTok{1}\NormalTok{, rmvnorm, }\DataTypeTok{n=}\DecValTok{1}\NormalTok{, }\DataTypeTok{sigma=}\NormalTok{sigma2}\OperatorTok{*}\NormalTok{I))}

\CommentTok{#use h to select from Z1 and Z2}
\NormalTok{Z=}\KeywordTok{matrix}\NormalTok{(}\KeywordTok{rep}\NormalTok{(}\DecValTok{0}\NormalTok{, }\DecValTok{2}\OperatorTok{*}\NormalTok{n), }\DataTypeTok{nrow =}\NormalTok{ n)}
\ControlFlowTok{for}\NormalTok{ (i }\ControlFlowTok{in} \DecValTok{1}\OperatorTok{:}\NormalTok{n) \{}
\NormalTok{Z[i,]=Z1[i,]}\OperatorTok{*}\NormalTok{h[i,}\DecValTok{1}\NormalTok{]}\OperatorTok{+}\NormalTok{Z2[i,]}\OperatorTok{*}\NormalTok{h[i,}\DecValTok{2}\NormalTok{]}
\NormalTok{\}}

\KeywordTok{plot}\NormalTok{(Z[,}\DecValTok{1}\NormalTok{], Z[,}\DecValTok{2}\NormalTok{])}
\end{Highlighting}
\end{Shaded}

\includegraphics{Conditional-Copula_files/figure-latex/unnamed-chunk-5-1.pdf}

\begin{Shaded}
\begin{Highlighting}[]
\NormalTok{dat=}\KeywordTok{data.frame}\NormalTok{(Z[,}\DecValTok{1}\NormalTok{], Z[,}\DecValTok{2}\NormalTok{], X[,}\DecValTok{2}\NormalTok{])}
\KeywordTok{colnames}\NormalTok{(dat)=}\KeywordTok{c}\NormalTok{(}\StringTok{"Y1"}\NormalTok{, }\StringTok{"Y2"}\NormalTok{, }\StringTok{"X"}\NormalTok{)}
\KeywordTok{write.table}\NormalTok{(dat, }\StringTok{"dat"}\NormalTok{)}

\CommentTok{#Get copula data}
\NormalTok{u=}\KeywordTok{t}\NormalTok{(}\KeywordTok{apply}\NormalTok{(Z, }\DecValTok{1}\NormalTok{, pnorm))}
\NormalTok{U=u[,}\DecValTok{1}\NormalTok{]}
\NormalTok{V=u[,}\DecValTok{2}\NormalTok{]}

\KeywordTok{par}\NormalTok{(}\DataTypeTok{mfrow=}\KeywordTok{c}\NormalTok{(}\DecValTok{1}\NormalTok{,}\DecValTok{3}\NormalTok{))}
\KeywordTok{plot}\NormalTok{(U[}\DecValTok{1}\OperatorTok{:}\NormalTok{(}\DecValTok{1}\OperatorTok{/}\DecValTok{3}\OperatorTok{*}\NormalTok{n)], V[}\DecValTok{1}\OperatorTok{:}\NormalTok{(}\DecValTok{1}\OperatorTok{/}\DecValTok{3}\OperatorTok{*}\NormalTok{n)], }\DataTypeTok{xlab =} \StringTok{'U'}\NormalTok{, }\DataTypeTok{ylab=}\StringTok{'V'}\NormalTok{, }\DataTypeTok{main =} \StringTok{'0<z<-1.5'}\NormalTok{)}
\KeywordTok{plot}\NormalTok{(U[(}\DecValTok{1}\OperatorTok{/}\DecValTok{3}\OperatorTok{*}\NormalTok{n)}\OperatorTok{:}\NormalTok{(}\DecValTok{2}\OperatorTok{/}\DecValTok{3}\OperatorTok{*}\NormalTok{n)], V[(}\DecValTok{1}\OperatorTok{/}\DecValTok{3}\OperatorTok{*}\NormalTok{n)}\OperatorTok{:}\NormalTok{(}\DecValTok{2}\OperatorTok{/}\DecValTok{3}\OperatorTok{*}\NormalTok{n)],}\DataTypeTok{xlab =} \StringTok{'U'}\NormalTok{, }\DataTypeTok{ylab=}\StringTok{'V'}\NormalTok{, }\DataTypeTok{main =} \StringTok{'1.5<z<3'}\NormalTok{)}
\KeywordTok{plot}\NormalTok{(U[(}\DecValTok{2}\OperatorTok{/}\DecValTok{3}\OperatorTok{*}\NormalTok{n)}\OperatorTok{:}\NormalTok{n], V[(}\DecValTok{2}\OperatorTok{/}\DecValTok{3}\OperatorTok{*}\NormalTok{n)}\OperatorTok{:}\NormalTok{n], }\DataTypeTok{xlab =} \StringTok{'U'}\NormalTok{, }\DataTypeTok{ylab=}\StringTok{'V'}\NormalTok{, }\DataTypeTok{main =} \StringTok{'3<z<5'}\NormalTok{)}
\end{Highlighting}
\end{Shaded}

\includegraphics{Conditional-Copula_files/figure-latex/unnamed-chunk-5-2.pdf}

\begin{Shaded}
\begin{Highlighting}[]
\CommentTok{#Set up marginal distribution}
\NormalTok{R=}\DecValTok{6}
\NormalTok{mu=}\KeywordTok{c}\NormalTok{(}\OperatorTok{-}\DecValTok{9}\NormalTok{, }\OperatorTok{-}\FloatTok{5.4}\NormalTok{, }\OperatorTok{-}\FloatTok{1.8}\NormalTok{, }\FloatTok{1.8}\NormalTok{, }\FloatTok{5.4}\NormalTok{, }\DecValTok{9}\NormalTok{)}
\NormalTok{sigmar=}\KeywordTok{rep}\NormalTok{(}\DecValTok{1}\OperatorTok{/}\KeywordTok{sqrt}\NormalTok{(}\DecValTok{10}\NormalTok{), R)}
\NormalTok{pi=}\KeywordTok{rep}\NormalTok{(}\DecValTok{1}\OperatorTok{/}\NormalTok{R, R)}
\NormalTok{theta=}\DecValTok{40}

\CommentTok{# evaluate the function at the point x, where the components }
\CommentTok{# of the mixture have weights w, means stored in u, and std deviations}
\CommentTok{# stored in s - all must have the same length.}
\NormalTok{F =}\StringTok{ }\ControlFlowTok{function}\NormalTok{(x,w,u,s) }\KeywordTok{sum}\NormalTok{( w}\OperatorTok{*}\KeywordTok{pnorm}\NormalTok{(x,}\DataTypeTok{mean=}\NormalTok{u,}\DataTypeTok{sd=}\NormalTok{s) )}

\CommentTok{#Marginal quantile function}
\CommentTok{# provide an initial bracket for the quantile. default is c(-1000,1000). }
\NormalTok{F_inv =}\StringTok{ }\ControlFlowTok{function}\NormalTok{(p,w,u,s,}\DataTypeTok{br=}\KeywordTok{c}\NormalTok{(}\OperatorTok{-}\DecValTok{1000}\NormalTok{,}\DecValTok{1000}\NormalTok{))}
\NormalTok{\{}
\NormalTok{  G =}\StringTok{ }\ControlFlowTok{function}\NormalTok{(x) }\KeywordTok{F}\NormalTok{(x,w,u,s) }\OperatorTok{-}\StringTok{ }\NormalTok{p}
  \KeywordTok{return}\NormalTok{( }\KeywordTok{uniroot}\NormalTok{(G,br)}\OperatorTok{$}\NormalTok{root ) }
\NormalTok{\}}


\CommentTok{#Get the orginal data}
\NormalTok{X1=}\KeywordTok{sapply}\NormalTok{(V, F_inv, }\DataTypeTok{w=}\NormalTok{pi, }\DataTypeTok{u=}\NormalTok{mu, }\DataTypeTok{s=}\NormalTok{sigmar)}
\NormalTok{X2=}\KeywordTok{sapply}\NormalTok{(U, F_inv, }\DataTypeTok{w=}\NormalTok{pi, }\DataTypeTok{u=}\NormalTok{mu, }\DataTypeTok{s=}\NormalTok{sigmar)}

\KeywordTok{par}\NormalTok{(}\DataTypeTok{mfrow=}\KeywordTok{c}\NormalTok{(}\DecValTok{1}\NormalTok{,}\DecValTok{3}\NormalTok{))}
\KeywordTok{plot}\NormalTok{(X1[}\DecValTok{1}\OperatorTok{:}\NormalTok{(}\DecValTok{1}\OperatorTok{/}\DecValTok{3}\OperatorTok{*}\NormalTok{n)], X2[}\DecValTok{1}\OperatorTok{:}\NormalTok{(}\DecValTok{1}\OperatorTok{/}\DecValTok{3}\OperatorTok{*}\NormalTok{n)],  }\DataTypeTok{xlab =} \StringTok{'X'}\NormalTok{, }\DataTypeTok{ylab=}\StringTok{'Y'}\NormalTok{, }\DataTypeTok{main =} \StringTok{'0<z<-1.5'}\NormalTok{)}
\KeywordTok{plot}\NormalTok{(X1[(}\DecValTok{1}\OperatorTok{/}\DecValTok{3}\OperatorTok{*}\NormalTok{n)}\OperatorTok{:}\NormalTok{(}\DecValTok{2}\OperatorTok{/}\DecValTok{3}\OperatorTok{*}\NormalTok{n)], X2[(}\DecValTok{1}\OperatorTok{/}\DecValTok{3}\OperatorTok{*}\NormalTok{n)}\OperatorTok{:}\NormalTok{(}\DecValTok{2}\OperatorTok{/}\DecValTok{3}\OperatorTok{*}\NormalTok{n)], }\DataTypeTok{xlab =} \StringTok{'X'}\NormalTok{, }\DataTypeTok{ylab=}\StringTok{'Y'}\NormalTok{, }\DataTypeTok{main =} \StringTok{'1.5<z<3'}\NormalTok{)}
\KeywordTok{plot}\NormalTok{(X1[(}\DecValTok{2}\OperatorTok{/}\DecValTok{3}\OperatorTok{*}\NormalTok{n)}\OperatorTok{:}\NormalTok{n], X2[(}\DecValTok{2}\OperatorTok{/}\DecValTok{3}\OperatorTok{*}\NormalTok{n)}\OperatorTok{:}\NormalTok{n], }\DataTypeTok{xlab =} \StringTok{'X'}\NormalTok{, }\DataTypeTok{ylab=}\StringTok{'Y'}\NormalTok{, }\DataTypeTok{main =} \StringTok{'3<z<5'}\NormalTok{)}
\end{Highlighting}
\end{Shaded}

\includegraphics{Conditional-Copula_files/figure-latex/unnamed-chunk-6-1.pdf}

\end{document}
